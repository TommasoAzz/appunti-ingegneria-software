\documentclass[../main]{subfiles}
\begin{document}

\section{Introduzione}
Un \g{progetto} è un insieme di quattro \g{attività} complesse:
\begin{itemize}
    \item \g{Pianificazione}: \g{attività} in cui si decide a priori, e meglio ancora dalla fine all’inizio per evitare di incorrere in mala pianificazione, il susseguirsi delle \g{attività} da svolgersi per completare il \g{progetto} che si sta per compiere. Per quanto riguarda un \g{compito} o un’\g{attività} è bene pianificare anche le risorse necessarie per portarli a termine: tempo, denaro, persone e strumenti;
    \item Analisi dei requisiti: \g{attività} in cui si definisce \textit{cosa} va fatto;
    \item Progettazione: \g{attività} in cui si definisce \textit{come} le \g{attività} pianificate e successivamente analizzate vanno fatte. Notare fin da subito che la progettazione qui si riferisce al verbo \textit{to design} e non al verbo \textit{to project};
    \item \g{Implementazione} (o realizzazione): come fare a rendere usabile ciò che è stato fino a questo momento progettato. È il momento in cui si passa veramente da un concetto astratto a uno concreto. Il termine \g{implementazione} rende molto meglio di realizzazione. È fondamentale però fare in modo che ciò che viene realizzato venga successivamente accertato della sua bontà. Questo viene fatto attraverso due \g{attività} dette di verifica e validazione.
\end{itemize}
Nel compiere un \g{progetto} bisogna fare attenzione a non concentrarsi sui soli dettagli inutili: in altre parole, potrebbe sembrare che una certa componente funzioni, ma non è detto che essa faccia ciò per cui è stata progettata (il che significa, \textit{non buttarsi a capofitto sul codice senza ragionare, non concentrarsi su dettagli implementativi per ore, che certamente non saranno quelli a creare la maggior parte dei problemi}, ma anche \textit{“non avere i polpastrelli fumanti, non fare la scimmia”}).\newline\newline
Ma per svolgere un \g{progetto} serve un ingrediente principale: questo ingrediente si chiama \g{ingegneria}, in inglese \textit{engineering}. L’\g{ingegneria} non si pone quindi il problema di creare principi nuovi, ma di applicare quelli che già esistono. \textit{Non crea conoscenza, la attua}.\newline\newline
In particolare, per quanto riguarda la creazione di un \g{prodotto software} esiste l’\g{ingegneria del software}. L’\g{ingegneria del software} richiede che il lavoro sia svolto con \g{efficacia} ed \g{efficienza}, e che entrambe siano misurabili (qualità oggettive). Tutto ciò che non è misurabile (soggettivo), all’\g{ingegneria del software} (in questo caso) non interessa.\newline\newline
Il \g{prodotto software} in \g{ingegneria del software} passa necessariamente attraverso un \g{ciclo di vita (del prodotto software)} seguendo delle \g{best practice}. Il \g{ciclo di vita (del prodotto software)} non ha sempre la stessa durata: tendenzialmente esso è lungo se il software viene usato (\textit{tanti utenti contenti}), breve se invece non è usato. Infatti, dopo il suo sviluppo, c’è il tempo di manutenzione, che è continua se gli utenti usano il software e richiedono nuove funzionalità e/o segnalano errori o malfunzionamenti da correggere.\newline\newline
L’\g{ingegneria del software} ha bisogno dell’informatica (\textit{computer science}), della matematica discreta, della ricerca operativa, della psicologia, della statistica, dell’economia.\newline\newline
La manutenzione può essere di due tipi:
\begin{itemize}
    \item Correttiva (“\textit{bad news}”): correzioni di emergenza o posticipate, in inglese \textit{deferred};
    \item Preventiva (“\textit{good news}”): programmata, in inglese \textit{time based}, con lo scopo di scovare errori, sulla base di verifica di condizioni, predittiva oppure basata su rischi (che potrebbero verificarsi).
\end{itemize}
\textit{“Il software non deperisce, cambia l’ambiente circostante (cioè la tecnologia). La manutenzione deve costare il meno possibile ed essere facile.”}\newline\newline
L’\g{ingegneria del software} nasce nel 1968 durante una conferenza NATO. Il motivo per cui è nata è stato raccogliere, organizzare e consolidare la conoscenza, in un \textit{body of knowledge}, per realizzare progetti software con la massima \g{efficienza} ed \g{efficacia}, applicando principi provenienti dall’ingegneria.\newline\newline
Il pensiero computazione, alla base e l’essenza dell’informatica, è fatto di:
\begin{itemize}
    \item Pensiero algoritmico: specifica una sequenza chiara e ordinata di passaggi;
    \item Pensiero logico: elaborazione di un ragionamento chiaro e preciso per convincere gli altri e convincere sé stessi della correttezza di un concetto;
    \item Astrazione: concentrarsi sulle informazioni rilevanti;
    \item Riconoscimento di elementi ricorrenti: trovare in un problema, o il problema stesso, un pattern già visto o già risolto che permetta di evitare di \textit{reinventare la ruota}.
\end{itemize}
Un \g{progetto} si affronta in gruppo, con spirito di \g{team work}.

\end{document}