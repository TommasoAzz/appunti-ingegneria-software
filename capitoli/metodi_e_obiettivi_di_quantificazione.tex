\documentclass[../main]{subfiles}
\begin{document}

\section{Metodi e obiettivi di quantificazione}
Misurare è importante perché permette di dare un valore a un'\g{attività} o ad un \g{compito} che si svolge, ovvero "\textit{si misura per essere quantificabili}". Permette di avere dei dati oggettivi su cui basarsi.\newline
L'atto di misurare va fatto con intelligenza: quando misurare diventa l'obiettivo da raggiungere, sostituisce il motivo per cui si sta misurando, e cessa di essere una buona misura.\newline
Effettuare \g{misurazioni} oggettive permette di essere:
\begin{itemize}
    \item Ripetibili;
    \item Confrontabili;
    \item Confidenti.
\end{itemize}
Basti pensare ai test sul software, la cui ragion d'essere è basata su questi principi. Ogni misurazione ha però i suoi limiti, con cui bisogna fare i conti: nelle realtà fisiche è l'approssimazione, nel software (in questo caso), è l'astrazione.
\subsection{Come misurare}
Effettuare misurazioni ogni tanto può fornire un'idea sullo stato attuale dell'oggetto sotto misura ma non offre molto di più. Affinché le misurazioni siano utili è bene che esse siano ripetute nel tempo e che i risultati siano utilizzati per fornire una serie storica da confrontare con valori di riferimento (per esempio provenienti da \g{best practice} di dominio). Lo scopo di questa operazione è tenere sotto controllo l'andamento e cercare proattivamente di migliorarlo, evitando (nel caso del software) la manutenzione correttiva ma passando a una preventiva.
\end{document}