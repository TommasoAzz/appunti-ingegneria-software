\documentclass[../main]{subfiles}
\begin{document}

\section{Verifica e validazione: Analisi dinamica}
Al concetto di analisi dinamica che, come sappiamo, richiede il \g{prodotto software} in esecuzione, il concetto di test.
I test si assicurano che l'oggetto sotto verifica (un'\g{unità}, un insieme di componenti) faccia quello che effettivamente dichiara. Essendo le combinazioni di possibili esecuzioni infinite, è bene ridurle ad un insieme finito.\newline
Elenchiamo alcune caratteristiche che contraddistinguono i test:
\begin{itemize}
    \item Devono essere portati avanti di pari passo con lo sviluppo (o si rischia di non volerli più fare), bilanciando il numero dei test da scrivere e lo sforzo per implementarli;
    \item Hanno delle esigenze che vanno tenute in considerazione nella progettazione del sistema (prima fra tutte quella di dichiarare le dipendenze);
    \item Ogni test deve specificare i valori di ingresso, di uscita (attesi) e lo stato iniziale del sistema;
\end{itemize}
\subsection{Tolleranza ai problemi}
La tolleranza ai problemi, anche detta in inglese \textit{fault tolerance}, è la capacità di saper distinguere i problemi fra errori umani (in inglese \textit{mistake}), la loro manifestazione (in inglese \textit{hardware/software fault}) e il risultato del problema (in inglese \textit{failure}) e di quanto (valore oggettivo) non è corretto (in inglese \textit{error}).
\begin{itemize}
    \item Si parla di \textit{failure} quando il comportamento del sistema devia da quanto ci si aspetta;
    \item Si parla di \textit{error} quando il risultato dell'esecuzione di un sistema non è corretto;
    \item Ciò che porta ad un \textit{error} è \textit{fault}.
\end{itemize}
I sistemi sono composizioni gerarchiche di componenti che a loro volta sono sistemi, più piccoli. Quindi un \textit{fault} è possibile che sia generato da un \textit{failure} di un sistema più grande.
\subsection{Criteri guida}
Oltre alle caratteristiche precedentemente indicate sui test, ve ne sono altre:
\begin{itemize}
    \item Oggetto della prova: può essere un'unità (\textit{test di unità}), un insieme di unità coordinate funzionalmente (\textit{test d'integrazione}), il sistema nel suo complesso (\textit{test di sistema});
    \item Obiettivo della prova: deve essere chiaro per ogni test che viene implementato (tipicamente corrisponde al nome del test, che deve essere sempre "parlante");
    \item Un test deve far parte del Piano di Qualifica (ovvero nel documento in cui viene descritto come il gruppo di \g{progetto} intende perseguire gli obiettivi di qualità);
    \item "\textit{Il test è buono se è cattivo}": se i test che scrivo passano tutti non è assicurata l'assenza di \textit{failure}, ma se il test non passa allora certamente si è davanti ad un \textit{fault} (la condizione è ovviamente che il test sia corretto);
    \item Se un test è difficile da implementare è possibile che la componente sotto test sia troppo complessa o troppo accoppiata;
    \item I test devono essere molteplici, rapidi, \textit{ripetibili}.
\end{itemize}
\subsection{Limiti dei test}
Come è stato detto, non esiste un modo per generare un test ideale. Esistono alcuni teoremi, ad esempio i teoremi di Howden e Weyuker, che dimostrano che non esiste un algoritmo che generi i test ideali, affidabili e validi, e che non è possibile verificare che dato un input, avvenga un dato output.\newline
Inoltre, anche Dijkstra con le sue tesi afferma che "un test di un programma può rilevare la presenza di un malfunzionamento, ma non la sua assenza".
\subsection{Gli elementi di un test}
\begin{itemize}
    \item Test case:
    \begin{itemize}
        \item Oggetto di prova (unità, insieme di unità, sistema);
        \item Input richiesto;
        \item Output richiesto;
        \item Ambiente d'esecuzione;
        \item Stato iniziale;
        \item Passi di esecuzione.
    \end{itemize}
    \item Test suite: insieme di test raggruppati con uno specifico criterio;
    \item Automazione dei test (pe eseguirli, registrare i risultati e analizzarli);
    \item Oracolo: metodo per generare a priori un risultato accettabile e confrontarlo con i risultati ottenuti.
\end{itemize}
L'oracolo può essere ottenuto dalle specifiche funzionali oppure "calcolando" i possibili risultati oppure usando componenti indipendenti terze che forniscono risultati assunti corretti.
\subsubsection{Classi di equivalenza}
Non essendo possibile effettuare un test per ogni possibile input, è necessario dividere i possibili input in classi di equivalenza, ovvero classi di valori che condividono le stesse caratteristiche, e che quindi è sufficiente testare un campione per classe per assicurarsi a sufficienza di coprire tutte le possibili situazioni.
\end{document}