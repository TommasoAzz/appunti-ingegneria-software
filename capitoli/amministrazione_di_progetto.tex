\documentclass[../main]{subfiles}
\begin{document}

\section{Amministrazione di progetto}
L'amministrazione di un \g{progetto} è l'insieme di tutte le \g{attività} che vengono fatte per organizzare e gestire l'ambiente di lavoro e di produzione.
L'amministrazione è un'attività compiuta principalmente dall'amministratore di concerto con i responsabili aziendali e/o di progetto.
\subsection{Attività dell'amministrazione}
L'amministrazione di un \g{progetto} consiste nelle seguenti attività:
\begin{itemize}
    \item Redazione delle norme, ovvero di tutte le regole e procedure per poter lavorare in modo efficace ed efficiente. Viene deciso il \g{way of working};
    \item Organizzazione e gestione delle risorse informatiche, ovvero tutto ciò che riguarda gli ambienti di lavoro, l'infrastruttura, gli strumenti e la documentazione.
\end{itemize}
\subsection{Documentazione di progetto}
In un \g{progetto} software è utile tenere traccia di tutto ciò che viene fatto e che è utile per misurare l'andamento dello stesso, per fornire conoscenza e mostrare dati oggettivi. I dati raccolti, le norme, il \g{way of working}, fa tutto parte della documentazione di un progetto.\newline
La documentazione di un progetto include informazioni sui \g{processi} di \g{progetto} e sui \g{prodotti} da realizzare. Per i primi sono inclusi contratti oltre che pianificazioni e consuntivi; per i secondi sono incluse le specifiche del cliente, i diagrammi di progettazione, i commenti al codice, i manuali, i piani di qualifica.
Essa, per essere realmente utile, deve essere \textit{sempre disponibile}. Non solo, deve essere:
\begin{itemize}
    \item Sempre disponibile;
    \item Chiaramente identificata (ogni documento deve avere il proprio nome specifico);
    \item Corretta nel contenuto;
    \item Verificata e approvata (senza l'approvazione può ancora essere considerato una bozza, ma una volta verificato e approvato, da un responsabile, esso certifica la sua correttezza);
    \item Aggiornata, datata e versionata;
    \item Con una lista di distribuzione, ovvero le persone interessate a ricevere e visionare il documento.
\end{itemize}
\subsection{Ambiente di lavoro}
Luogo, reale o virtuale, e strumenti con cui un lavoro viene svolto. L'ambiente è formato dalle persone, dai ruoli, dalle procedure e dall'infrastruttura.
È importante che l'infrastruttura sia adeguata al lavoro che deve essere svolto. L'ambiente di lavoro in generale deve essere:
\begin{itemize}
    \item Completo, con tutto il necessario per lavorare;
    \item Ordinato, tutto ciò di cui c'è bisogno è facilmente identificabile e utilizzabile;
    \item Aggiornato, ciò che è obsoleto non deve essere d'intralcio ma va tenuto ai fini del versionamento e del tracciamento dello storico del lavoro.
\end{itemize}
Gli strumenti utili alla gestione del progetto sono molteplici e hanno più scopi:
\begin{itemize}
    \item Pianificazione e controllo dei costi (per esempio, strumenti che generano diagrammi di Gantt, come Microsoft Project);
    \item Collaborazione, divisione del lavoro, coordinamento attività (per esempio, Issue Tracking System (ITS), come Atlassian Jira);
    \item Gestione documentale;
    \item Versionamento;
    \item Configurazione.
\end{itemize}
Gli strumenti utili alla gestione del \g{processo} di sviluppo sono invece:
\begin{itemize}
    \item Raccolta, classificazione e tracciamento dei requisiti;
    \item Progettazione di diagrammi (per esempio, diagrammi UML di classi, casi d'uso, architettura);
    \item Codifica, automazione del processo di build, testing, integrazione continua, versionamento, analisi statica del codice.
\end{itemize}
\subsection{Configurazione}
Un \g{prodotto software} è l'insieme ordinata di parti unite da regole quali specifiche, programmi, verifiche, manuali. Il software tipicamente ha una propria configurazione che è separata dallo sviluppo del programma e va gestita separatamente.
Fanno parte della configurazione le seguenti attività:
\begin{itemize}
    \item Identificazione di configurazione, che corrisponde ad identificare tutti i \textit{Configuration Item (CI)} che compongono il prodotto, formati da un ID, un nome, una data di creazione, un autore, un registro delle modifiche e uno stato corrente;
    \item Controllo di baseline, corrispondente a gestire le \textit{baseline}, ossia un insieme di \textit{CI} consolidato in un dato istante. A una baseline corrisponde tipicamente un punto fermo su di una linea temporale, chiamato \textit{milestone}. Un \g{progetto} consiste in una successione di milestone, pianificate a priori, a cui sono collegate una o più baseline. La loro gestione garantisce il controllo dell'avanzamento del \g{progetto} stesso;
    \item Controllo di \g{versione}, corrispondente al salvataggio dello stato di uno o più CI senza rischiare sovrascritture (\textit{check-out}), in maniera collaborativa e senza conflitti (\textit{check-in$\rightarrow$commit}), salvando il proprio contributo su di un repository, che può essere sia locale che distribuito e quindi remoto;
    \item Gestione delle modifiche, corrispondente al coordinamento delle richieste di modifiche da parte di utenti, sviluppatori o legate alla competizione che portano a dover analizzare delle proposte di modifiche, che viene trattata come se fosse un'attività da svolgere.
\end{itemize}
Le \g{milestone} vanno pianificate, e nel seguente ordine:
\begin{itemize}
    \item \g{Way of working}: milestone da fissare fin da subito per definirlo con chiarezza;
    \item Analisi dei requisiti: una volta fissati i requisiti si può iniziare a compiere il lavoro in parallelo, ma fino a quel momento bisogna definirli insieme;
    \item Discussioni e revisioni con il committente/proponente: fissare dei punti in cui venga verificato l'andamento dello sviluppo del \g{prodotto} e se rispetta le aspettative del proponente (non va assunta la correttezza, va chiesto un feedback per accertarsi);
    \item Prototipo o Proof of Concept: punto in cui si deve avere una bozza che dimostra il prodotto (prototipo come baseline anticipata) o una parte funzionante del prodotto che ne mostra i punti principali in maniera corretta (Proof of Concept come baseline incrementale).
\end{itemize}
Una modifica, ma anche una funzionalità da aggiungere:
\begin{itemize}
    \item Deve essere richiesta (\textit{Change Request}) da qualcuno, che ne specifica il motivo e l'urgenza;
    \item Va analizzata, inclusa la fattibilità,
    \item Va accettata o scartata;
    \item Se accettata, va gestita secondo il \g{way of working} di progetto o aziendale, tracciandone lo stato di avanzamento (con l'ausilio ad esempio di ITS) fino alla sua chiusura.
\end{itemize}
\end{document}