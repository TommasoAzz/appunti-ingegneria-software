\documentclass[../main]{subfiles}
\begin{document}

\section{Ciclo di vita del software}
Un \g{prodotto software} ha un proprio \g{ciclo di vita}:
\begin{itemize}
    \item Concepimento: il software viene pensato da qualcuno (per opportunità personale e/o presenza di stakeholder);
    \item Sviluppo: il software viene concretizzato (da idea a concretizzazione);
    \item Utilizzo: il software realizzato viene utilizzato (possibilmente \textit{"tanti utenti contenti"});
    \item Ritiro: il software non viene più utilizzato (nessuno lo usa più o non è più conveniente continuare a mantenerlo: \textit{discounted}).
\end{itemize}
Riprendendo il concetto di \g{ciclo di vita} come macchina a stati, la transizione fra uno stato ed un altro avviene mediante l'esecuzione di \g{attività} di \g{processi}.\newline
Parliamo di \g{fase} quando intendiamo un segmento temporale continuo, con un inizio ed una fine, in cui il software si trova in uno stato del \g{ciclo di vita} o in una transizione fra stati.
\subsection{Modelli di ciclo di vita}
Esistono più tipologie di \g{modelli}, per questo si parla di essi al plurale. Esistono, ad esempio, \g{modelli} utili per software che deve essere costantemente in funzione e quindi mantenuto, e \g{modelli} per software che una volta usato può essere ritirato dal mercato poiché il suo scopo è finito.
\subsubsection{Non-modello "Code'n'fix"}
È uno dei primi modi di lavorare ai \g{prodotti software}. Non si può dire che è stato ideato, perché veniva adottato e basta prima che si inventasse l'\g{ingegneria del software} nel 1968. Il codice sorgente veniva scritto e se andava bene si teneva, altrimenti veniva corretto.
\subsubsection{Modello sequenziale - a cascata - waterfall}
L'idea chiave di questo \g{modello} è avere processi ripetibili.\newline
Si basa su una successione rigida e strettamente sequenziale di \g{fasi} (ognuna avente delle chiare pre-condizioni in ingresso e altrettanto chiare post-condizioni in uscita verso la \g{fase} successiva), senza ritorni a \g{fasi} precedenti (perché tornare indietro significherebbe ripetere l'intera \g{fase}).\newline
I prodotti di questo \g{modello} sono principalmente documenti:
\begin{itemize}
    \item Il \g{prodotto software} diventa quasi una conseguenza della produzione di documentazione, poiché avviene verso la fine del \g{ciclo di vita};
    \item La loro approvazione permette di cambiare \g{fase} (vengono raggiunte le post-condizioni);
    \item Ogni documento dimostra il soddisfacimento delle pre-condizioni e delle post-condizioni della \g{fase}.
\end{itemize}
Le fasi sono assolutamente distinte, non sovrapposte, e vengono definite con:
\begin{itemize}
    \item Attività pianificate (con pre-condizioni e post-condizioni);
    \item Documenti prodotti;
    \item Responsabilità e ruoli coinvolti;
    \item Scadenze.
\end{itemize}
Tutto ciò porta il modello sequenziale ad essere particolarmente adatto in situazioni in cui il sistema organizzativo è molto complesso, infatti è stato importato nel mondo dello sviluppo software da quello dell'industria.\newline
Schema del \g{modello} (secondo lo standard ISO/IEC 12207):
\begin{enumerate}
    \item Analisi dei requisiti: è un'attività del processo di sviluppo da cui posso uscire solo se i progettisti hanno compreso appieno i documenti di analisi prodotti;
    \item Progettazione: è un'attività nuovamente del processo di sviluppo in cui viene realizzato il design dell'architettura software e del sistema e si esce solo se tutti i casi sono stati progettati e i programmatori sanno esattamente cosa fare;
    \item Realizzazione: viene codificato il software, testato e successivamente collaudato;
    \item Manutenzione: comporta ritorni alle fasi precedenti.
\end{enumerate}
Ma il \g{modello} non è esente da critiche:
\begin{itemize}
    \item Se capitano errori bisogna ricominciare da capo (e l'uomo fallisce);
    \item Se l'analisi dei requisiti è poco chiara (perché il cliente è stato poco chiaro o perché magari nemmeno lui sa bene quello che vuole) si rischia di progettare un prodotto che non è quello richiesto;
    \item È troppo rigido e ha una visione troppo burocratica e irrealistica.
\end{itemize}
Sono possibili alcune correzioni:
\begin{itemize}
    \item Prototipazione: vengono creati dei \g{prototipi} usa e getta per capire meglio requisiti e/o le possibili soluzioni;
    \item Cascata con ritorni: viene aggiunta la possibilità di tornare indietro. Ma i ritorni sono \g{incrementi} o \g{iterazioni}? Se in un ritorno è possibile per uno stakeholder ripensare a una parte del sistema, e potenzialmente rimuoverla, si parla di \g{iterazione}. Se invece è solo possibile aggiungere, possibilmente piccole parti, si parla di \g{incremento}.
\end{itemize}
Come consiglio generale, è bene evitare di posticipare troppo l'integrazione fra componenti o si rischia una \textit{"big-bang integration"} (che è l'esatto opposto della \textit{"continuous integration"}).\newline
Ritornando su \g{incrementi} e \g{iterazioni}:
\begin{itemize}
    \item I modelli incrementali possono produrre valore ad ogni incremento, abbastanza rapidamente, dando priorità alle funzionalità fondamentali;
    \item I modelli iterativi sono applicabili in ogni tipologia di modello di \g{ciclo di vita}, con gli opportuni vincoli, e sono ottimi se ci sono limiti ben definiti di azione altrimenti si rischia di non convergere alla soluzione del problema.
\end{itemize}
\subsubsection{Modello incrementale}
È un \g{modello} basato su rilasci multipli e successivi, in cui i requisiti da soddisfare per realizzare il sistema sono classificati secondo l'importanza strategica che viene ad essi attribuita: vengono prima definiti dei punti fermi, poi quelli di contorno per poter via via avere un sistema più stabile.
Lo schema generale di un modello incrementale si propone in questo modo:
\begin{enumerate}
    \item Definizione dei requisiti; % 1
    \item Assegnazione dei requisiti agli incrementi; % 2
    \item Progettazione dell'architettura del sistema (fino a questo punto il sistema è stabile); % 3
    \item Sviluppo di un \g{incremento} del sistema; % 4
    \item Validazione dell'\g{incremento}; % 5
    \item Integrazione dell'\g{incremento}; % 6
    \item Validazione del sistema e ritorno al punto 4 se mancano \g{incrementi} altrimenti si è in possesso del sistema finale. % 7
\end{enumerate}
I primi tre punti vengono attraversati una volta sola, per avere una base solida per i successivi cicli di sviluppo (punti 4-7). Quindi i requisiti e l'architettura del sistema sono definiti stabilmente una volta sola, mentre lo sviluppo degli \g{incrementi} è ciclico.
Schema del \g{modello} (secondo lo standard ISO/IEC 12207):
\begin{enumerate}
    \item Analisi dei requisiti e progettazione architetturale;
    \item Progettazione di dettaglio;
    \item Realizzazione e ritorno al punto 2 se mancano \g{incrementi} altrimenti si è in possesso del sistema finale.
\end{enumerate}
\subsubsection{Modello evolutivo}
Questa tipologia di \g{modello di ciclo di vita del software} si adatta bene a contesti in cui i requisiti non sono chiari fin da subito e i problemi devono necessariamente essere affrontati man mano. È utile anche quando c'è la necessità di avere contemporaneamente il rilascio e la manutenzione di più versioni (per esempio: un browser). Di conseguenza, certi stati del \g{ciclo di vita} vengono necessariamente attraversati più volte (ad esempio, la sequenza $analisi\;preliminare \rightarrow analisi\;e\;sviluppo\;singola\;evoluzione \rightarrow rilascio\;versione$). Ogni \g{fase} può avere quindi iterazioni multiple e addirittura parallele.
Schema del \g{modello} (secondo lo standard ISO/IEC 12207):
\begin{enumerate}
    \item Analisi preliminare;
    \item Consolidamento, analisi e progettazione;
    \item Realizzazione, in cui si ottiene un \g{prototipo} o il prodotto finale dell'evoluzione, tornando poi al punto 2.
\end{enumerate}
\subsubsection{Modello a componenti}
Originato con l'idea che molto di ciò che c'è da fare è già stato fatto e che quindi è possibile riutilizzarlo (l'analisi dei requisiti è quindi pensata e adattata al riuso di software già presente). L'idea alla base è il \g{riuso} sistematico di componenti pre-esistenti proprie oppure di terze parti.
\subsubsection{Metodi agili}
Sono metodi di lavoro nati in contrapposizione ai rigidi modelli introdotti (e presenti fino alla fine degli anni '90).
I principi fondanti sono:
\begin{itemize}
    \item Individui e interazione rispetto ai processi e agli strumenti (l'eccesiva rigidità ostacola l'emergere di valore);
    \item Software funzionante rispetto a documentazione comprensiva (la documentazione non spesso corrisponde a software funzionante). Osservazione: la documentazione non è formata solo da commenti, deve essere completa anche di spiegazioni alle scelte progettuali;
    \item Collaborazione con i clienti rispetto a negoziazione tramite contratti (l'interazione con gli stakeholder va incentivata e non ingessata);
    \item Risposte al cambiamento e alle modifiche rispetto a seguire un piano (capacità di adattamento). Osservazione: senza avere un piano di rischi e avanzamenti non si prosegue. Ben venga l'adattamento, ma dipende dai costi.
\end{itemize}
L'idea alla base dei metodi agili è la \textit{user story}, ossia la funzionalità richiesta dal cliente: è una definizione di ciò che deve fornire come funzionalità il sistema ed un riassunto delle conversazioni con gli stakeholder per capire il problema.\newline
Tipicamente, nei metodi agili si fanno piccoli incrementi con un valore aggiunto che possono essere sviluppati separatamente in maniera continua e successivamente integrati al loro completamento.
L'obiettivo è:
\begin{itemize}
    \item Avere sempre qualcosa da mostrare al cliente;
    \item Avere avanzamenti che sono progressi reali;
    \item Rendere soddisfatti gli sviluppatori;
    \item Dimostrare che il prodotto software è ben integrato e verificato.
\end{itemize}
Alcuni esempi di questi metodi sono SCRUM (spiegato di seguito) e Kanban.
\subsubsection{SCRUM}
SCRUM è un metodo agile. Si può descrivere brevemente nel seguente modo:
\begin{itemize}
    \item Le user story (dei metodi agili) raccontano le richieste del cliente;
    \item Il responsabile del progetto (lo SCRUM Master) decide l'ordine (un elenco non statico) in cui fare le cose per realizzare le richieste del cliente. Questo elenco è chiamato \textit{product backlog};
    \item La divisione del lavoro in \textit{sprint} (periodi di breve durata, 2-4 settimane) in cui ci sono sottoinsiemi di attività da fare scelti tra quelli disponibili nel \textit{product backlog}, chiamati \textit{sprint backlog}. Al suo interno, uno \textit{sprint} è così composto:
    \begin{itemize}
        \item \textit{Sprint Planning}: vengono selezionate le attività da svolgere dal \textit{product backlog} per inserirle nello \textit{sprint backlog};
        \item \textit{Daily SCRUM}: incontro giornaliero di 15 minuti, in piedi, in cui ci si parla dei problemi riscontrati, del lavoro svolto e del lavoro che si farà. Non si risolvono problemi;
        \item \textit{Sprint Review} e \textit{Sprint Retrospective}: rispettivamente incontri di breve durata (un paio d'ore ciascuno) in cui si discute rispettivamente del \g{prodotto software} realizzato e del controllo di qualità.
    \end{itemize}
\end{itemize}
\subsubsection{SEMAT}
Questo modello si differenzia dagli altri precedentemente presentati perché è diviso in temi, in cui c'è un livello base iniziale e uno finale che è l'obiettivo da raggiungere, passando per più step intermedi.
\begin{itemize}
    \item Stakeholder, ovvero le persone con cui ho a che fare per realizzare il sistema software:
    \begin{itemize}
        \item Riconosciuti (so chi sono gli stakeholder);
        \item Rappresentati (li rappresento);
        \item Partecipanti (sono partecipanti attivi del progetto);
        \item D'accordo (sono nella stessa linea di pensiero con i miei stakeholder);
        \item Soddisfazione nello sviluppo;
        \item Soddisfazione nell'uso;
    \end{itemize}
    \item Opportunità, ovvero le circostanze che rendono sensato il progetto:
    \begin{itemize}
        \item Identificate;
        \item \dots
        \item Accumulo di beneficio (svolgere il progetto ha portato un plusvalore);
    \end{itemize}
    \item Requisiti, ovvero le caratteristiche da soddisfare per considerare corretto il lavoro svolto:
    \begin{itemize}
        \item Concepiti;
        \item \dots
        \item Soddisfatti;
    \end{itemize}
    \item Sistema software:
    \begin{itemize}
        \item Selezionata l'architettura;
        \item Dimostrabile (ho una buona architettura e corretta per gli scopi);
        \item Usabile;
        \item Pronto (dopo l'implementazione);
        \item Operativo (svolge le attività/azioni richieste dal cliente);
        \item Ritirato (a meno che non debba continuare ad esistere);
    \end{itemize}
    \item Gruppo:
    \begin{itemize}
        \item \textit{Seeded} (faccio ciò che serve per selezionare le persone);
        \item Formato (il team è formato);
        \item Collaborazione (il team collabora);
        \item Performante (il team è efficace e efficiente);
        \item Aggiornato (il team può aggiornarsi, il lavoro è finito).
    \end{itemize}
    \item Lavoro:
    \begin{itemize}
        \item Inizializzato (lavoro definito);
        \item Preparato (tutto è pronto per essere svolto);
        \item Iniziato (lavoro cominciato);
        \item Sotto controllo;
        \item Concluso (lavoro finito);
        \item Terminato (il lavoro è finito e si è sicuri che tutto è stato correttamente svolto);
    \end{itemize}
    \item Way of working:
    \begin{itemize}
        \item Principi fissati;
        \item Elementi fondanti fissati;
        \item Principi in uso;
        \item Principi in ordine (le regole vengono rispettate);
        \item Lavoro bene (lavorare bene e essere soddisfatti deve essere misurabile);
        \item Ritirato (il \g{way of working} definito non ha più utilità a lavoro finito).
    \end{itemize}
\end{itemize}
L'obiettivo finale è essere \g{sistematici}, \g{disciplinati} e \g{quantificabili}.
\end{document}